\documentclass[12pt]{article}
\usepackage[margin=1in]{geometry}
\usepackage{amsmath}
\usepackage{amssymb}
\usepackage{amsthm}

\theoremstyle{plain}
\newtheorem{thm}{Theorem}
\newtheorem{cor}{Corollary}
\newtheorem{prop}{Proposition}

\theoremstyle{definition}
\newtheorem{defn}{Definition}

\newcommand{\Lop}{\operatorname{\mathcal{L}}}
\newcommand{\im}{\operatorname{im}}
\newcommand{\Span}{\operatorname{span}}
\newcommand{\sgn}{\operatorname{sgn}}
\newcommand{\adj}{\operatorname{adj}}

\title{Notes on determinants and the classical adjoint}
\author{Jonathan Lin}
\date{\today}

\begin{document}
\maketitle

\begin{prop}
For any square matrix $A$, we have that $\det{A} = \det{A^T}$, where $A^T$ is the transpose of $A$.
\end{prop}
\begin{proof}
If we let $A(x, y)$ denote the entry of $A$ in the $x$-th row and the $y$-th column, then we have that $A^T(i, \sigma(i)) = A(\sigma(i), i)$ by definition. Hence
\begin{align*}
	\det{A^T} &= \sum_{\sigma \in S_n}(\sgn{\sigma})A(\sigma(1), 1)\cdots A(\sigma(n), n) \\
		&= \sum_{\sigma \in S_n}(\sgn{\sigma})A(1, \sigma^{-1}(1))\cdots A(n, \sigma^{-1}(n)) \\
		&= \det{A}
\end{align*}
so we are done.
\end{proof}

\begin{prop}
Row addition does not change the determinant.
\end{prop}
\begin{proof}
Use the fact that the determinant is multilinear, then alternating.
\end{proof}

\begin{prop}
Suppose $A \in M_{r \times r}(F)$, $C \in M_{s \times s}(F)$, $B \in M_{r \times s}(F)$. Then
\[\det \begin{bmatrix} A & B \\ 0 & C\end{bmatrix} = \det(A)\det(C).\]
\end{prop}
\begin{proof}
	Define the function $M(A, B, C)$ as the determinant of the above block matrix. By the properties of the determinant, $M$ is alternating and multilinear as a function of the rows of $C$. Hence $M(A, B, C) = \det(C)M(A, B, I)$. Using row reduction $M(A, B, I) = M(A, 0, I)$. Now $M(A, 0, I)$ is clearly alternating and multilinear as a function of the columns of $A$. Then $M(A, 0, I) = \det(A)M(I, 0, I) = \det(A)$. We conclude that $M(A, B, C) = \det(A)\det(C)$ and we are done.
\end{proof}

Define the cofactor matrix $A(i | j)$ to be the matrix obtained from $A$ by deleting the $i$th row and $j$th column. Let $C_{ij} = (-1)^{i + j}\det(A(i|j))$. It follows that $\det{A} = \sum_{i = 1}^n A_{ij}C_{ij}$ from the usual cofactor definition of the determinant.

\begin{defn}
The classical adjoint of $A$ is defined to be the transpose of the matrix of cofactors. That is,
\[ \adj(A)(i, j) = C_{ji}.\]
\end{defn}

We have the identity 
\[\sum_{i = 1}^nA_{ik}C_{ij} = \delta_{jk}\det{A} \]
because the left hand side describes the determinant of the matrix $A$ if the $j$th column was replaced by the $k$th column.
\end{document}
