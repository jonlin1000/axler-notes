In these notes we have introduced (one implicitly, one explicitly) two different bases for the vector space of polynomials $P[x]$. First we have the standard monomial basis
\[1, x, x^2, x^3, \dots\]
and then we introduced the falling factorial basis
\[1, (x), (x)_2, (x)_3, \dots\]
where 
\[(x)_k = x(x-1)(x-2)\cdots(x-k+1).\]

It is clear that both sets of polynomials form bases. Moreover, due to the nature of the monomial basis, the change of basis ``matrix'' (in the sense of an infinite matrix) is upper triangular (so working with it we have no analytic difficulties). It makes sense to consider the coefficients of a change of basis formula. We first define $s(n, k)$ to be the coefficients such that 
\[(x)_n = \sum_{k = 0}^{\infty}s(n,k)x^k\] is true. Note that $s(n,k) = 0$ if $k > n$. Similarly, we can write
\[x^n = \sum_{k = 0}^{\infty}S(n,k)(x)_k.\] Note as well that $S(n, k) = 0$ if $k > n$. If we substitute the first equation into the second we get

\begin{align*}
x^n &= \sum_{k = 0}^{\infty}S(n, k)(x)_k \\
&= \sum_{k = 0}^{\infty}S(n, k)\sum_{j = 0}^{\infty}s(k, j)x^j \\
&= \sum_{j = 0}^{\infty}\left[\sum_{k = 0}^{\infty}S(n, k)s(k, j)\right]x^j.
\end{align*}

By linear independence it follows that 
\[\sum_{k = 0}^{\infty}S(n, k)s(k, j) = \begin{cases}1 & n = j \\ 0 & \text{otherwise}\end{cases}.\]

This is perhaps not so surprising on its own: This actually always happens when you consider any change of basis from one basis back to itself. What will be surprising is that $s$ and $S$ have combinatorial meaning. For example, when we derived a formula for the Bell Numbers, we realized $S(n, k)$ as the number of ways of partitioning $\{1, \dots, n\}$ into exactly $k$ disjoint subsets. But what about $s(n, k)$?

First, we must observe that $s(n, k)$ is not always positive. It has precisely the same sign as $(-1)^{n-k}$, and we call it a \textit{signed Stirling number of the first kind}. On this note we will denote $c(n, k)$ to be the coefficient of $x^k$ in the polynomial 
\[r_n(x) = x(x + 1)(x + 2)(\cdots)(x+n-1).\]

\begin{proposition}
For all positive integers $n, k$, 
\[c(n,k) = c(n-1, k-1) + (n-1)c(n-1, k).\]
\end{proposition}
\begin{proof}
From the fact that
\[r_n(x) = r_{n-1}(x)(x + n-1) = xr_{n-1}(x) + (n-1)r_{n-1}(x)\]
this fact easily follows by degree counting.
\end{proof}
 
\begin{theorem}
Let $S_n$ be the symmetric group on $\{1, 2, \dots, n\}$. Then $c(n, k)$ equals the number of elements of $S_n$ with exactly $k$ cycles.
\end{theorem}
\begin{proof}
Let $d(n,k)$ be the number of elements of $S_n$ with exactly $k$ cycles. We will show that $d(n, k)$ follows the same recurrence as $c(n, k)$ with the same initial conditions. 

When $n = 0, k = 0$ we have that $c(n, k) = 1$. When $k = 0, n \neq 0$ we have that $c(n, k) = 0$. Similarly, when $n = 0, k = 0$ there is only one permutation with $0$ cycles, which is the empty permutation, so $d(0, 0) = 1$. If $n \neq 0$ every permutation has at least one cycle so that $d(n, 0) = 1$, as desired.

Suppose that $\sigma$ is any permutation of $[n-1]$ with $k$ cycles. Then there are $n-1$ ways to add the element $n$ into an existing cycle (We add $n$ to a cycle such that $\tilde{\sigma}(n) = i$ for each $1 \leq i \leq n-1$. Also, there's a unique way to create a unique cycle consisting of just the element $n$ in any permutation of $[n-1]$ with $k-1$ cycles. This describes every permutation of $[n]$ with $k$ cycles. It follows that 
\[d(n,k) = (n-1)d(n-1, k) + d(n-1, k-1).\]
\end{proof}

Now we will discuss the \textit{Stirling number of the second kind} $S(n, k)$.
\begin{proposition}
We have for all $n$ and $k$ that $S(n, k) = S(n-1, k-1) + kS(n-1, k)$.
\end{proposition}
\begin{proof}
Given a partition of $[n]$ into $k$ blocks, removing $n$ gives a partition of $[n-1]$ either with $k$ blocks or with $k-1$ blocks. Conversely, every partition of $[n-1]$ into $k-1$ blocks gives rise to a partition of $[n]$ into $k$ blocks (by adding $\{n\}$) and every partition of $[n-1]$ into $k$ blocks gives rise to $k$ partitions of $[n]$ into $k$ blocks. The recurrence is seen to hold.
\end{proof}

Here is a claim about the generating function of the Stirling numbers of the second kind:

\begin{proposition}
For $k \geq 0$ we have that
\[\sum_{n \geq k}S(n-k)x^n = \frac{x^k}{(1-x)(1-2x)\cdots(1-kx)}.\]
\end{proposition}
\begin{proof}
Sketch: The proposition is true for $k = 1$.
If we denote $S_k(x) = \sum_{n \geq k}S(n-k)x^n$, we can use the recurrence relation to show that 
\[S_k(x) = xS_{k-1}(x) + kxS_k(x).\] Rewriting this as
\[S_k(x) = \frac{x}{1-kx}S_{k-1}(x)\] the inductive hypothesis will prove the claim.
\end{proof}

As a product of simple looking generating functions the above proposition might have a simpler combinatorial proof. We will see how this will work out in the next section.

\subsection{The Combinatorial Meaning of Stirling Orthogonality}

Above we deduced the formula
\[\sum_{k = 0}^{\infty}S(n, k)s(k, j) = \begin{cases}1 & n = j \\ 0 & \text{otherwise}\end{cases}\]
For general $n$, $m$, and $k$. In this subsection we will show that this formula has combinatorial meaning.

To show the sum is $0$, we will use the following principle: suppose that $S$ is a set with a weight function on $S$ and there is some involution $\Phi: S \to S$ which satisfies $W(\Phi(s)) = -W(s)$ for all $s \in S$. Then
\[\sum_{s \in S}W(s) = 0.\]

To show Stirling orthogonality, then, we will find some sign reversing involution between a combinatorial object.

\begin{definition}
A \textbf{gadget} of type $(n, m, k)$ (where $n \geq m \geq k$) is a tuple consisting of a partition of $[n]$ into $m$ blocks $B_1, \dots, B_m$ and a permutation of $\{B_1, \dots, B_m\}$ into $k$ cycles.
We will define the weight of such a gadget to be $(-1)^{m-k}$.
\end{definition}

So in this way, gadgets realize the Stirling Numbers of both kinds in one setup. With the weight we will find a specific sign reversing involution; then we will be done.

First, we observe that the sum of the weights of all the $(n, m, k)$ gadgets is $S(n, m)s(m, k)$. This follows from the fact that the weight of any $(n,m,k)$ gadget is specified by the sign of the stirling number $s(m, k)$. For the sign-reversing involution, we say a cycle in the gadget $\gamma$ is \textit{ample} if it involves more than one element of $[n]$, that is, either the cycle contains greater than $1$ block or it contains a block with more than $1$ element.

Since $n > k$ there is at least one ample cycle in $\gamma$. Let $x$ be the smallest element of $[n]$ that lies in an ample cycle. If $x$ is in a block with other elements of $[n]$ rip $x$ off to form a new block of size $n$, and put it before the old block of the cycle. For example, if $\gamma$ is of the form
\[\gamma = \dots(\dots\{x, a, b, \dots\}\dots)\dots\] then we rip off $x$ to create the cycle
\[\gamma' = \dots(\dots\{x\}\{a, b, \dots\}\dots)\dots.\] Otherwise, if $x$ is in a block by itself, then merge $\{x\}$ with whatever follows it in its cycle.

Then this defines a sign reversing involution, for it changes the number of blocks by $1$, so the weight of each will be negative. So we're done in this case.